\documentclass[12pt]{article}
\usepackage[italian]{babel}
\usepackage[utf8x]{inputenc}
\usepackage{amsmath}
\usepackage{graphicx}
\usepackage[colorinlistoftodos]{todonotes}

\begin{document}
	\begin{titlepage}
			\newcommand{\HRule}{\rule{\linewidth}{0.5mm}} % Defines a new command for the horizontal lines, change thickness here
		
		\center % Center everything on the page
		
		%----------------------------------------------------------------------------------------
		%	HEADING SECTIONS
		%----------------------------------------------------------------------------------------
		
		\textsc{\LARGE Università degli studi di Padova}\\[1.5cm] % Name of your university/college
		\includegraphics[scale=0.3]{images/unipd_logo.png}\\[1cm] % Include a department/university logo - this will require the graphicx package
		\textsc{\Large Relazione progetto per il corso di Web Information Management}\\[0.5cm] % Major heading such as course name
		\textsc{\large Corso di Laurea in Informatica, A.A. 2016-2017}\\[0.5cm] % Minor heading such as course title
		%----------------------------------------------------------------------------------------
		%	TITLE SECTION
		%----------------------------------------------------------------------------------------
		
		\HRule \\[0.4cm]
		{ \huge \bfseries Analisi usabilità strumentimusicali.net}\\[0.4cm] % Title of your document
		\HRule \\[1.5cm]
			\begin{minipage}{0.4\textwidth}
				\begin{flushleft} \large
					\emph{Studente:}\\
				 Matteo Slanzi \\ \#1100866
				\end{flushleft}
			\end{minipage}
			~
			\begin{minipage}{0.4\textwidth}
				\begin{flushright} \large
					\emph{Docente:} \\
					Massimo Marchiori
				\end{flushright}
			\end{minipage}\\[2cm]
	\end{titlepage}
	\section{Indice}
	\newpage
	\section{Elenco delle figure}
	\newpage
	\section{Introduzione}
	\vspace {0.5cm}
	Il progetto riguardante il corso di Web Information Management consiste nell'analisi di usabilità di un sito web. La mia scelta riguarda un sito di e-commerce di strumenti musicali, che vende principalmente in Italia, una vasta gamma di strumenti e accessori musicali, sia professionali che amatoriali. 
	\\
	Il sito mi è sembrato fin da subito ben fatto e ho deciso di analizzarlo anche per via di un acquisto di un paio di cuffie, fatto qualche mese fa.
	\newpage
	\section{Analisi delle pagine}
		\subsection{Homepage}
	La homepage del sito è molto ricca, nell'header troviamo subito il logo del sito, il menu, la barra di ricerca e lo slideshow che mostra gli articoli in sconto o novità. Più sotto troviamo gli articoli in evidenza, gli ultmi arrivi del mese, ultime recensioni e news. In fondo, il footer raggruppa diverse informazioni quali, l'assistenza post vendita, le modalità di pagamento e consegna, diritti e privacy, link ai vari social e tutte le informazione dell'azienda che si occupa della gestione e vendita dei prodotti del sito.
\end{document}
